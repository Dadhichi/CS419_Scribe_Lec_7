%
% This is the LaTeX template file for lecture notes for CS294-8,
% Computational Biology for Computer Scientists.  When preparing 
% LaTeX notes for this class, please use this template.
%
% To familiarize yourself with this template, the body contains
% some examples of its use.  Look them over.  Then you can
% run LaTeX on this file.  After you have LaTeXed this file then
% you can look over the result either by printing it out with
% dvips or using xdvi.
%
% This template is based on the template for Prof. Sinclair's CS 270.

\documentclass[11pt, twosides]{article}
\usepackage[utf8]{inputenc}
\usepackage{graphicx}
\usepackage{graphics}
\usepackage{amsmath}
\usepackage{amsfonts}
\usepackage{amssymb}
\usepackage{amsthm}
\usepackage{xcolor}
\usepackage{tikz}
\usepackage{pgfplots}
\usepackage{hyperref}
\usepackage{float}
\setlength{\oddsidemargin}{0.25 in}
\setlength{\evensidemargin}{-0.25 in}
\setlength{\topmargin}{-0.6 in}
\setlength{\textwidth}{6.5 in}
\setlength{\textheight}{8.5 in}
\setlength{\headsep}{0.75 in}
\setlength{\parindent}{0 in}
\setlength{\parskip}{0.1 in}

%
% The following commands set up the lecnum (lecture number)
% counter and make various numbering schemes work relative
% to the lecture number.
%
\newcounter{lecnum}
\renewcommand{\thepage}{\thelecnum-\arabic{page}}
\renewcommand{\thesection}{\thelecnum.\arabic{section}}
\renewcommand{\theequation}{\thelecnum.\arabic{equation}}
\renewcommand{\thefigure}{\thelecnum.\arabic{figure}}
\renewcommand{\thetable}{\thelecnum.\arabic{table}}

%
% The following macro is used to generate the header.
%
\newcommand{\lecture}[4]{
%   \pagestyle{myheadings}
   \thispagestyle{plain}
   \newpage
   \setcounter{lecnum}{#1}
   \setcounter{page}{1}
   \noindent
   \begin{center}
   \framebox{
      \vbox{\vspace{2mm}
    \hbox to 6.28in { {\bf CS 419M Introduction to Machine Learning
                        \hfill Spring 2021-22} }
       \vspace{4mm}
       \hbox to 6.28in { {\Large \hfill Lecture #1: #2  \hfill} }
       \vspace{2mm}
       \hbox to 6.28in { {\it Lecturer: #3 \hfill Scribe: #4} }
      \vspace{2mm}}
   }
   \end{center}
   \markboth{Lecture #1: #2}{Lecture #1: #2}
}

%
% Convention for citations is authors' initials followed by the year.
% For example, to cite a paper by Leighton and Maggs you would type
% \cite{LM89}, and to cite a paper by Strassen you would type \cite{S69}.
% (To avoid bibliography problems, for now we redefine the \cite command.)
% Also commands that create a suitable format for the reference list.
% \renewcommand{\cite}[1]{[#1]}
% \def\beginrefs{\begin{list}%
%         {[\arabic{equation}]}{\usecounter{equation}
%          \setlength{\leftmargin}{2.0truecm}\setlength{\labelsep}{0.4truecm}%
%          \setlength{\labelwidth}{1.6truecm}}}
% \def\endrefs{\end{list}}
% \def\bibentry#1{\item[\hbox{[#1]}]}

%Use this command for a figure; it puts a figure in wherever you want it.
%usage: \fig{NUMBER}{SPACE-IN-INCHES}{CAPTION}
% \newcommand{\fig}[3]{
% 			\vspace{#2}
% 			\begin{center}
% 			Figure \thelecnum.#1:~#3
% 			\end{center}
% 	}
% Use these for theorems, lemmas, proofs, etc.
\newtheorem{theorem}{Theorem}[lecnum]
\newtheorem{lemma}[theorem]{Lemma}
\newtheorem{proposition}[theorem]{Proposition}
\newtheorem{claim}[theorem]{Claim}
\newtheorem{corollary}[theorem]{Corollary}
\newtheorem{definition}[theorem]{Definition}
% \newenvironment{proof}{{\bf Proof:}}{\hfill\rule{2mm}{2mm}}

% **** IF YOU WANT TO DEFINE ADDITIONAL MACROS FOR YOURSELF, PUT THEM HERE:

\begin{document}
%FILL IN THE RIGHT INFO.
%\lecture{**LECTURE-NUMBER**}{**DATE**}{**LECTURER**}{**SCRIBE**}
\lecture{7}{Classification}{Abir De}{Group 2}
%\lecture{x}{Title}{Abir De}{Group y}


\section{Topics covered in lecture}
\label{Sec:1}
\begin{enumerate}
    \item What is classification task
    \item What are the classsification models
    \item Support Vector Machine (not Covered in lecture)
\end{enumerate}

\section{Classification task}
\label{Sec:2}
\begin{itemize}
\item Given training set $\mathcal{D} = \{(x_i,y_i) \mid y_i \in \mathcal{G}\}$, $\mathcal{G} = \{y_1,y_2\}$ , $x_i \in \mathbb{R}^d$, find $m(x) \mapsto y$.
\item Test set $= \{x_i \in \mathbb{R}^d\}$, $y_i$ are not known in test set.
\end{itemize}

\section{Probabilistic Approch}
\label{Sec:3}
\begin{align*}
   & P_m(y \mid x) = \frac{1}{1+e^{-w^Tx\cdot y}} \\
  \implies  &\max_{w} \prod_{i \in \mathcal{D}} P_m(y_i \mid x_i) \\
  \implies  &\max_{w} \sum_{i \in \mathcal{D}} \log{P_m(y_i \mid x_i)}\\
\implies    &\min_{w} \sum_{i \in mathcal{D}} \log{\left(1+e^{w^Tx_i \cdot y_i}\right)}
\end{align*}


\section{Simpler way to classify}
\label{Sec:4}
    Let $\mathcal{G} = \{+1,-1\}$, for some other labels we can convert that to {+1,-1}.
    We want some linear boundary to classify given points into $\mathcal{G}$.
    \begin{align*}
        &w^Tx+b \ge \Delta, \quad y=1\\
        &w^Tx+b \le -\Delta,\quad y=-1\\
        &\min_w f(w)
    \end{align*}
where $f(w)$ is any regularizer.
This problem can be solved when there is \hyperref[fig:no overlapping]{no overlapping} between the classification variables  but there isn't a solution for something like \hyperref[fig:overlapping]{This}

\begin{figure}[H]
       \centering 
    \begin{tikzpicture}[
        declare function={a(\x)=0.9*\x-1.7;},
        declare function={b(\x)=0.9*\x-0.9;}
    ]{label:2}
    \begin{axis}[
        domain=0:5,
        axis lines=middle,
        axis equal image,
        xtick=\empty, ytick=\empty,
        enlargelimits=true,
        clip mode=individual, clip=false
    ]
    % \addplot [red, only marks, mark=*, samples=300, mark size=0.75]
    %     {0.5*(a(x)+b(x)) + 0.5*rand*(a(x)-b(x))};
    \addplot [thick] {a(x)};
    \addplot [thick] {b(x)};
    \pgfmathsetseed{08022022}
    \foreach \p in {1,...,250}
    {
        \fill[red]    (200+rand*70,450+75*rand ) circle (5);
    }
    \foreach \p in {1,...,250}
    {
        \fill[blue]    (500+rand*70,250+rand*75) circle (5);
    }
    
    \end{axis}
    \end{tikzpicture}
        \label{fig:no overlapping}
        \caption{no overlapping}
    \end{figure}
    
\begin{figure}[H]
    \centering
    \begin{tikzpicture}[
        declare function={a(\x)=0.9*\x-1.7;},
        declare function={b(\x)=0.9*\x-0.9;}
    ]{label:2}
    \begin{axis}[
        domain=0:5,
        axis lines=middle,
        axis equal image,
        xtick=\empty, ytick=\empty,
        enlargelimits=true,
        clip mode=individual, clip=false
    ]
    \addplot [red, only marks, mark=*, samples=50, mark size=0.75]
         {0.5*(a(x)+b(x)) + 0.5*rand*(a(x)-b(x))};
    \addplot [blue, only marks, mark=*, samples=50, mark size=0.75]
         {0.5*(a(x)+b(x)) + 0.5*rand*(a(x)-b(x))};
 \addplot [thick] {a(x)};
 \addplot [thick] {b(x)};
    \pgfmathsetseed{08022022}
    \foreach \p in {1,...,250}
    {
        \fill[red]    (150+rand*70,300+75*rand ) circle (5);
    }
    \foreach \p in {1,...,250}
    {
        \fill[blue]    (400+rand*70,250+rand*75) circle (5);
    }
    
    \end{axis}
    \end{tikzpicture}
        \label{fig:overlapping}
        \caption{Overlapping}
\end{figure}


We can ignore the overlapping points,\\
Let ${I}^+ = \{ i \mid y_i = 1 \}$, ${I}^- = \{ i \mid y_i = -1 \}$,  ${S}^+ \in {I}^+$,${S}^- \in{I}^-$ such that $|{S^+} \cup {S^-}| = n $,
\begin{equation*}
    \min_{\zeta_i} { f(w) - \left( \sum_{i \in S^+} \mathbb{I}\left( w^Tx_i + b \ge \Delta \right) + \sum_{i \in S^-} \mathbb{I} \left( w^Tx_i + b \le \Delta \right) \right)}
\end{equation*}


\section{Adding Slack variable to solve the overlapping case}
\label{Sec:5}
Modifying optimisation problem to include overlapping points,
\begin{align*}
    w^Tx_i + b &\ge \Delta - \zeta_i, \, \, \, \, y_i = 1\\
    w^Tx_i +b &\le -\Delta + \zeta_i,\, \, \, \, y_i = -1\\
        y_i \cdot \left(w^Tx_i + b \right) &\ge \Delta - \zeta_i \\
\zeta_i &\ge 0
\end{align*}
with above conditions, we have to solve following
\begin{align*}
    &\min_{w,b,\zeta_i} C\sum_{i \in \mathcal{D}} \zeta_i
 + \lambda \mid \mid w \mid \mid ^2 
 \end{align*}


\section{Group Details and Individual Contribution}
% Fill this part
\begin{table}[H]
    \centering
    \begin{tabular}{|c|c|c|}
     \hline
        Name & Roll number & contribution \\
        \hline
        Dadhichi Telwadkar & 20D070083 & \ref{Sec:1},\ref{Sec:2},\ref{Sec:3},\ref{Sec:4},\ref{Sec:5}\\
    \hline
    \end{tabular}
    % \caption{Caption}
    \label{tab:my_label}
\end{table}
\end{document}





